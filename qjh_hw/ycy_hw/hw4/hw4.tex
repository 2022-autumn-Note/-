\documentclass{article}

\usepackage{amsmath}
\usepackage{multirow}
\usepackage{color}
\usepackage{geometry}

\renewcommand{\thesection}{\arabic{section}.}
\renewcommand{\thesubsection}{(\arabic{subsection}).}

\geometry{top=3cm, bottom=3cm}
\pagestyle{empty}

\begin{document}
\section{}
If the PPP holds,
\[
    \varepsilon = e \frac{P}{P^\ast} = 1
    \;\Longrightarrow\;
    e = \frac{P^\ast}{P}.
\]
Employing the formula, we can solve the Price and the Predicted column. However, for the actual exchange rate, it lacks the information about the real exchange rate.

\begin{table}[h]
\centering
\begin{tabular}{|c|c|c|c|c|}
    \hline
    \multirow{2}*{Country} & \multirow{2}*{Currency} & \multirow{2}*{Big-Mac Price} & \multicolumn{2}{c|}{Exchange rate (per US dollar)} \\
    \cline{4-5}
     &  &  & Predicted (PPP) & Actual \\
    \hline
    USA & Dollar & 5 & \color{blue}1 & \color{blue}1 \\
    \hline
    China & Yuan & 20 & \color{blue}4 & 7 \\
    \hline
    Japan & Yen & \color{blue}375 & 75 & 100 \\
    \hline
    UK & Pound & 4 & 0.8 & \color{blue}? \\
    \hline
\end{tabular}
\end{table}

\section{}
\subsection{}
As $Y$, $T$ and $r$ are exogenous variables, we firstly compute
\begin{align*}
    C &= 1000 + \frac{3}{4} (Y-T) = 6400, \\
    I &= 1200 - 100 r^\ast = 700.
\end{align*} 
Then we have
\begin{align*}
    S &= Y - C - G = 600, \\
    \text{Excess Savings} &= S - I = -100, \\
    X &= \text{Excess Savings} = -100.
\end{align*}

\subsection{}
To achieve the equilibrium, 
\[
    500 - 200 \varepsilon = X(\varepsilon) = \text{Excess Savings}
    \;\Longrightarrow\;
    \varepsilon = 3.
\]

\subsection{}
With $G$ rising to 1200 and $Y$ and $T$ unchanged, 
\begin{align*}
    S_{ng} &= Y - C - T = 800, \\
    S & = Y - C - G = 400, \\
    \text{Excess Savings} &= S - I = -300, \\
    X &= \text{Excess Savings} = -300.
\end{align*}
Again, we can compute that
\[
    500 - 200 \varepsilon = X(\varepsilon) = \text{Excess Savings}
    \;\Longrightarrow\;
    \varepsilon = 4.
\]
\end{document}