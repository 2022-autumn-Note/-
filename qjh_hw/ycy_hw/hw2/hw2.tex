\documentclass{article}
\usepackage{amsmath}

\pagestyle{empty}
\renewcommand{\thesubsection}{(\arabic{subsection}).}

\begin{document}
\section{}
\subsection{}
Plug in the market price and the consumption,
\[
    \text{CPI for 2016} =
    \frac{6 \times 15 + 3 \times 8}{5 \times 15 + 2 \times 8} \times 100 = 125.27 .
\]

\subsection{}
\[
    \text{GDP Deflator for 2016} = 
    \frac{6 \times 25 + 3 \times 15}{5 \times 25 + 2 \times 15} \times 100 = 125.81 .
\]

\subsection{}
\[
    \text{GDP Deflator for 2015} = 
    \frac{5 \times 20 + 2 \times 10}{6 \times 20 + 3 \times 10} \times 100 = 80 .
\]

\section{}
Let $M$ denotes the married population, $S$ the single population and $P_A = M+R$ the adult population. In this question, the divorce rate is $d = 0.02$ and the marriage rate is $r = 0.03$. When the percentage of single people in the adult population is steady, we have
\[
    Md = Sr
    \, \Longrightarrow \,
    \frac{P_A - S}{P_A} d = \frac{S}{P_A} r
    \, \Longrightarrow \,
    \frac{S}{P_A} = \frac{d}{d + r}
    = \frac{2}{5} = 40 \%. 
\]
That is, the steady-state percentage of
single people in the adult population is $40 \%$.

\section{}
\subsection{}
According to the definition, the labor force consists of group 1, 2, 3, 4, and 6. Adding the up, we know that
\[
    \text{Labor Force} = 65 .
\]
The labor force and group 5, 7, 8 make up the adults, so
\[
    \text{Labor Force Participation Rate}
    = \frac{65}{65 + 25} = \frac{13}{18}.
\]

\subsection{}
The umemployed is the group 4. Therefore,
\[
    \text{Unemployment Rate} = \frac{10}{65} = \frac{2}{13}.
\]

\end{document}